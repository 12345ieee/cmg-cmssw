\chapter{Results}

{\color{magenta}{HERE SLIDES:7,9,10,11,30}}

\begin{figure}[h!]
  \begin{center}
    \includegraphics[width=1.5\textwidth]{RESULTS/RecoilMTMETshape.pdf}
    \caption{
      Impact of the recoil correcgion for the MT (left) MET (right).
      The range used used in the Wmass fit is indicated by the arrows.
    }
    \label{fig:appliedRecoil}
  \end{center}
%\end{figure}                                                                                                                                                              
%\begin{figure}[h!]                                                                                                                                                        
  \begin{center}
    \includegraphics[width=.45\textwidth]{RESULTS/WmassFit_StatUnc.pdf}
    \includegraphics[width=.45\textwidth]{RESULTS/WmassFit_ResolU2Unc.pdf}
    \includegraphics[width=.45\textwidth]{RESULTS/WlikemassFit_StatUnc.pdf}
    \caption{
      Top Left: Wmass statistical uncertainty; Top Right: W mass systematic uncertainty.
      Bottom: statistical uncertainties for the Wlike system.
    }
    \label{fig:resultsFit}
  \end{center}
\end{figure}

The W mass is obtained from a maximum likelihood fit of pT,MET,MT templates generated at discrete values of MW with $\Gamma$W fixed at the Standard Model value. The recoi\
l correction alter the MC templates to get closer to the description of the DATA.
The impact of the recoil corrections for the W+jets MC signal sample is shown in Figure~\ref{fig:appliedRecoil} for the MT (left) and MET (right).

The largest errors are statistical in nature, both from the statistics of the W sample and also the statistics of the Z samples which are used to define the systematic un\
certainties i.e. due in the recoil correction.

%To extract the Wmass different template are built with different Wmass hyphostesis.                                                                                       
To extract the statistical uncertainties on the Wmass fit, we compare
pseudodata with nominal corrections and different templates nominal correction.
The results are shown in Figure~\ref{fig:resultsFit} (left).
To extract the systematic uncertainties on the Wmass fit due to the recoil correction, we compare:
pseudodata with nominal corrections and different “up” and “down” templates based on the uncertainties on the fitted recoil parameters.
The results are shown in Figure~\ref{fig:resultsFit} (right).
We see that the systematcs errors are of the same order or lower than the statistical error.

\begin{figure}[h!]
  \begin{center}
    \includegraphics[width=1.5\textwidth]{RESULTS/MTresidual.pdf}
    \caption{
     MT residual (defined as the difference between central correction and up/down variation)
    }
    \label{fig:mtResidual}
  \end{center}
\end{figure}

To validate the Wmass fit results, we look at the event by event MT residual defined as the difference between central correction and up/down variation. The residuals are\
 shown in Figure~\ref{fig:mtResidual}.
The Mean is shifted by 5 MeV and is compatible with the value we find from the maximum likelihood fit of the Wmass with the MT.

\begin{figure}[h!]
  \begin{center}
    \includegraphics[width=0.45\textwidth]{RESULTS/trend_MT_WlikePos.pdf}
    \includegraphics[width=0.45\textwidth]{RESULTS/trend_MET_WlikePos.pdf}
    \includegraphics[width=0.45\textwidth]{RESULTS/trend_MT_WPos.pdf}
    \includegraphics[width=0.45\textwidth]{RESULTS/trend_MET_WPos.pdf}
    \includegraphics[width=0.45\textwidth]{RESULTS/trend_MT_WNeg.pdf}
    \includegraphics[width=0.45\textwidth]{RESULTS/trend_MET_WNeg.pdf}
    \caption{
     Trend of the systematics uncertainties due to the recoil as function of number of sigma shown separately for the scale (red), U1 resolution (blu) and U2 resolution (\
green).
     We present the MT (left) and MET (right) for the Wlike (top), W positive (middle) and W negative (bottom).
    }
    \label{fig:uncScan}
  \end{center}
\end{figure}

In Figure~\ref{fig:uncScan}, the trend of the systematics uncertainties on the W boson mass due to the recoil modelling as function of number of sigma of the recoil uncer\
tainties is veryfied.
The effect is shown for $W^{+}$, $W^{-}$ and $W_{like}^{+}$ separately and for MT and MET separately.
The magnitude on MET is more pronounced than in MT as expected.
For the MT the effect on the $W^{+}$ is more important than in the $W^{-}$, this is expected given that the $W^{+}$ tend to be more forward and the uncertaintes are large\
r for the forward part.
For the MT The scale has a minor impact respect the resolution in all the cases and the two resolutions are comparables.
When comparing the Wlike vs the W system, we see that the Wlike uncertainties are slightly lower than the one for the Wlike, this is likely due to the fact that the Wlike\
 system is not re-weighted for the rapidity and boson pt distribution.

The uncertainties associated with the measurements are listed in Table XXX.
